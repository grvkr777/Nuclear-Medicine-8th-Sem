\documentclass{article}
\usepackage[utf8]{inputenc}

\title{Update-5}
\author{Gaurav Kar}
\date{18111025}

\begin{document}

\maketitle

\section{In Continuation}
Nuclear medicine tests differ from most other imaging modalities in that diagnostic tests primarily show the physiological function of the system being investigated as opposed to traditional anatomical imaging such as CT or MRI. Nuclear medicine imaging studies are generally more organ-, tissue- or disease-specific (e.g.: lungs scan, heart scan, bone scan, brain scan, tumor, infection, Parkinson etc.) than those in conventional radiology imaging, which focus on a particular section of the body (e.g.: chest X-ray, abdomen/pelvis CT scan, head CT scan, etc.). In addition, there are nuclear medicine studies that allow imaging of the whole body based on certain cellular receptors or functions. Examples are whole body PET scans or PET/CT scans, gallium scans, indium white blood cell scans, MIBG and octreotide scans.\\While the ability of nuclear metabolism to image disease processes from differences in metabolism is unsurpassed, it is not unique. Certain techniques such as fMRI image tissues (particularly cerebral tissues) by blood flow and thus show metabolism. Also, contrast-enhancement techniques in both CT and MRI show regions of tissue that are handling pharmaceuticals differently, due to an inflammatory process.

Diagnostic tests in nuclear medicine exploit the way that the body handles substances differently when there is disease or pathology present. The radionuclide introduced into the body is often chemically bound to a complex that acts characteristically within the body; this is commonly known as a tracer. In the presence of disease, a tracer will often be distributed around the body and/or processed differently. For example, the ligand methylene-diphosphonate (MDP) can be preferentially taken up by bone. By chemically attaching technetium-99m to MDP, radioactivity can be transported and attached to bone via the hydroxyapatite for imaging. Any increased physiological function, such as due to a fracture in the bone, will usually mean increased concentration of the tracer. This often results in the appearance of a "hot spot", which is a focal increase in radio accumulation or a general increase in radio accumulation throughout the physiological system. Some disease processes result in the exclusion of a tracer, resulting in the appearance of a "cold spot". Many tracer complexes have been developed to image or treat many different organs, glands, and physiological processes.

\end{document}
