\documentclass{article}
\usepackage[utf8]{inputenc}

\title{Update-3}
\author{Gaurav Kar}
\date{18111025}

\begin{document}
 

\maketitle


Cholescintigraphy or hepatobiliary scintigraphy is scintigraphy of the hepatobiliary tract, including the gallbladder and bile ducts. The image produced by this type of medical imaging, called a cholescintigram, is also known by other names depending on which radiotracer is used, such as HIDA scan, PIPIDA scan, DISIDA scan, or BrIDA scan. Cholescintigraphic scanning is a nuclear medicine procedure to evaluate the health and function of the gallbladder and biliary system. A radioactive tracer is injected through any accessible vein and then allowed to circulate to the liver, where it is excreted into the bile ducts and stored by the gallbladder until released into the duodenum.\\Use of cholescintigraphic scans as a first-line form of imaging varies depending on indication. For example for cholecystitis, cheaper and less invasive ultrasound imaging may be preferred, while for bile reflux cholescintigraphy may be the 
best choice. Most radiotracers for cholescintigraphy are metal complexes of iminodiacetic acid (IDA) with a radionuclide, usually technetium-99m. This metastable isotope has a half-life of 6 hours, so batches of radiotracer must be prepared as needed using a moly cow. A widely recognized trade name for the preparation kits is TechneScan. These radiopharmaceuticals include the following:technetium Tc 99m lidofenin	hepatobiliary iminodiacetic acid; dimethyl-iminodiacetic acid	HIDA	An early and widely used tracer; not used as much anymore, as others have progressively replaced it, but the term "HIDA scan" is sometimes used even when another tracer was involved, being treated as a catch-all synonym.technetium Tc 99m iprofenin ,paraisopropyl-iminodiacetic acid	PIPIDA	
technetium Tc 99m, disofenin, diisopropyl-iminodiacetic acid	DISIDA	
technetium Tc 99m mebrofenin.Cholecystitis
The investigation is usually conducted after an ultrasonographic examination of the abdominal right upper quadrant for a patient presenting with abdominal pain. If the noninvasive ultrasound examination fails to demonstrate gallstones, or other obstruction to the gallbladder or biliary tree, in an attempt to establish a cause of right upper quadrant pain, a cholescintigraphic scan can be performed as a more sensitive and specific test.

Cholescintigraphy for acute cholecystitis has sensitivity of 97 percent specificity of 94 percent. Several investigators have found the sensitivity being consistently higher than 90 percent though specificity has varied from 73–99, yet compared to ultrasonography, cholescintigraphy has proven to be superior. The scan is also important to differentiate between neonatal hepatitis and biliary atresia, because an early surgical intervention in form of Kasai portoenterostomy or hepatoportoenterostomy can save the life of the baby as the chance of a successful operation after 3 months seriously decreases.
\end{document}
