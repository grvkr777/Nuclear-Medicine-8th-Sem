\documentclass{article}
\usepackage[utf8]{inputenc}

\title{Update-8}
\author{Gaurav Kar (18111025)}
\date{}

\begin{document}

\maketitle

\section{Continuation}
Elastography is a relatively new imaging modality that maps the elastic properties of soft tissue. This modality emerged in the last two decades. Elastography is useful in medical diagnoses, as elasticity can discern healthy from unhealthy tissue for specific organs/growths. For example, cancerous tumours will often be harder than the surrounding tissue, and diseased livers are stiffer than healthy ones. There are several elastographic techniques based on the use of ultrasound, magnetic resonance imaging and tactile imaging. The wide clinical use of ultrasound elastography is a result of the implementation of technology in clinical ultrasound machines. Main branches of ultrasound elastography include Quasistatic Elastography/Strain Imaging, Shear Wave Elasticity Imaging (SWEI), Acoustic Radiation Force Impulse imaging (ARFI), Supersonic Shear Imaging (SSI), and Transient Elastography.[15] In the last decade a steady increase of activities in the field of elastography is observed demonstrating successful application of the technology in various areas of medical diagnostics and treatment monitoring.\\PET is both a medical and research tool used in pre-clinical and clinical settings. It is used heavily in the imaging of tumors and the search for metastases within the field of clinical oncology, and for the clinical diagnosis of certain diffuse brain diseases such as those causing various types of dementias. PET: PET is a valuable research tool to learn and enhance our knowledge of the normal human brain, heart function, and support drug development. PET is also used in pre-clinical studies using animals. It allows repeated investigations into the same subjects over time, where subjects can act as their own control and substantially reduces the numbers of animals required for a given study. This approach allows research studies to reduce the sample size needed while increasing the statistical quality of its results.

Physiological processes lead to anatomical changes in the body. Since PET is capable of detecting biochemical processes as well as expression of some proteins, PET can provide molecular-level information much before any anatomic changes are visible. PET scanning does this by using radiolabelled molecular probes that have different rates of uptake depending on the type and function of tissue involved. Regional tracer uptake in various anatomic structures can be visualized and relatively quantified in terms of injected positron emitter within a PET scan.
\end{document}
