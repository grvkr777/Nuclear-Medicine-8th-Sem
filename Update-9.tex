\documentclass{article}
\usepackage[utf8]{inputenc}

\title{Update-9}
\author{Gaurav Kar (18111025)}
\date{}

\begin{document}

\maketitle

\section{In continuation}
What is nuclear medicine imaging?
Nuclear medicine imaging is a method of producing images by detecting radiation from different parts of the body after a radioactive tracer is given to the patient. The images are digitally generated on a computer and transferred to a nuclear medicine physician, who interprets the images to make a diagnosis.\\Radioactive tracers used in nuclear medicine are, in most cases, injected into a vein. For some studies, they may be given by mouth. These tracers aren’t dyes or medicines, and they have no side effects. The amount of radiation a patient receives in a typical nuclear medicine scan tends to be very low.\\Why would you need a nuclear imaging test?\\Nuclear imaging is used primarily to diagnose or treat illnesses. Conditions diagnosed by nuclear medicine imaging include:
\begin{itemize}
\item Blood disorders.
\item Thyroid disease, including hypothyroidism.
\item Heart disease.
\item Gallbladder disease.
\item Lung problems.
\item Bone problems, including infections or breaks.
\item Kidney disease, including infections, scars or blockages.
\item Cancer.
\end{itemize}
Nuclear medicine imaging can also be used to treat conditions or to evaluate how treatment is working. One example of this is radioimmunotherapy, which combines radiation and immunotherapy to deliver radiation precisely to a targeted area.\\TEST DETAILS\\How is nuclear medicine imaging different than other radiologic tests?
The main difference between nuclear medicine imaging and other radiologic tests is that nuclear medicine imaging evaluates how organs function, whereas other imaging methods assess anatomy (how the organs look).\\The advantage of assessing the function of an organ is that it helps physicians make a diagnosis and plan treatments for the part of the body being evaluated.\\Before the test\\There are no general rules for preparing for the nuclear medicine test, since each type of test has its own requirements.\\For example, one test may require you to not eat or drink - except for water - from six hours before the test until the test is complete. Another test may have no restrictions at all.


\end{document}
