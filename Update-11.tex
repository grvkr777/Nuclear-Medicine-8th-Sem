\documentclass{article}
\usepackage[utf8]{inputenc}

\title{Update-11}
\author{Gaurav Kar}
\date{}

\begin{document}

\maketitle

\section{In Continuation}
Myocardial perfusion imaging: Myocardial perfusion imaging (MPI) is a form of functional cardiac imaging, used for the diagnosis of ischemic heart disease. The underlying principle is that under conditions of stress, diseased myocardium receives less blood flow than normal myocardium. MPI is one of several types of cardiac stress test. A cardiac specific radiopharmaceutical is administered, e.g., 99mTc-tetrofosmin (Myoview, GE healthcare), 99mTc-sestamibi (Cardiolite, Bristol-Myers Squibb) or Thallium-201 chloride. Following this, the heart rate is raised to induce myocardial stress, either by exercise on a treadmill or pharmacologically with adenosine, dobutamine, or dipyridamole (aminophylline can be used to reverse the effects of dipyridamole). SPECT imaging performed after stress reveals the distribution of the radiopharmaceutical, and therefore the relative blood flow to the different regions of the myocardium. Diagnosis is made by comparing stress images to a further set of images obtained at rest which are normally acquired prior to the stress images.\\In the nuclear power sector, the SPECT technique can be applied to image radioisotope distributions in irradiated nuclear fuels. Due to the irradiation of nuclear fuel (e.g. uranium) with neutrons in a nuclear reactor, a wide array of gamma-emitting radionuclides are naturally produced in the fuel, such as fission products (cesium-137, barium-140 and europium-154) and activation products (chromium-51 and cobalt-58). These may be imaged using SPECT in order to verify the presence of fuel rods in a stored fuel assembly for IAEA safeguards purposes, to validate predictions of core simulation codes, or to study the behavior of the nuclear fuel in normal operation,  or in accident scenarios.\\Reconstruction: Reconstructed images typically have resolutions of 64×64 or 128×128 pixels, with the pixel sizes ranging from 3–6 mm. The number of projections acquired is chosen to be approximately equal to the width of the resulting images. In general, the resulting reconstructed images will be of lower resolution, have increased noise than planar images, and be susceptible to artifacts.\\Scanning is time-consuming, and it is essential that there is no patient movement during the scan time. Movement can cause significant degradation of the reconstructed images, although movement compensation reconstruction techniques can help with this. A highly uneven distribution of radiopharmaceutical also has the potential to cause artifacts. A very intense area of activity (e.g., the bladder) can cause extensive streaking of the images and obscure neighboring areas of activity. This is a limitation of the filtered back projection reconstruction algorithm. Iterative reconstruction is an alternative algorithm that is growing in importance, as it is less sensitive to artifacts and can also correct for attenuation and depth dependent blurring. Furthermore, iterative algorithms can be made more efficacious using the Superiorization methodology. Attenuation of the gamma rays within the patient can lead to significant underestimation of activity in deep tissues, compared to superficial tissues. Approximate correction is possible, based on relative position of the activity, and optimal correction is obtained with measured attenuation values. Modern SPECT equipment is available with an integrated X-ray CT scanner. As X-ray CT images are an attenuation map of the tissues, this data can be incorporated into the SPECT reconstruction to correct for attenuation. It also provides a precisely registered CT image, which can provide additional anatomical information. Scatter of the gamma rays as well as the random nature of gamma rays can also lead to the degradation of quality of SPECT images and cause loss of resolution. Scatter correction and resolution recovery are also applied to improve resolution of SPECT images.
\end{document}
