\documentclass{article}
\usepackage[utf8]{inputenc}

\title{Update-10}
\author{Gaurav Kar}
\date{}

\begin{document}

\maketitle

\section{In Continuation}
Single-photon emission computed tomography (SPECT, or less commonly, SPET) is a nuclear medicine tomographic imaging technique using gamma rays. It is very similar to conventional nuclear medicine planar imaging using a gamma camera (that is, scintigraphy), but is able to provide true 3D information. This information is typically presented as cross-sectional slices through the patient, but can be freely reformatted or manipulated as required.\\The technique needs delivery of a gamma-emitting radioisotope (a radionuclide) into the patient, normally through injection into the bloodstream. On occasion, the radioisotope is a simple soluble dissolved ion, such as an isotope of gallium(III). Most of the time, though, a marker radioisotope is attached to a specific ligand to create a radioligand, whose properties bind it to certain types of tissues. This marriage allows the combination of ligand and radiopharmaceutical to be carried and bound to a place of interest in the body, where the ligand concentration is seen by a gamma camera.\\Instead of just "taking a picture of anatomical structures", a SPECT scan monitors level of biological activity at each place in the 3-D region analyzed. Emissions from the radionuclide indicate amounts of blood flow in the capillaries of the imaged regions. In the same way that a plain X-ray is a 2-dimensional (2-D) view of a 3-dimensional structure, the image obtained by a gamma camera is a 2-D view of 3-D distribution of a radionuclide.\\SPECT imaging is performed by using a gamma camera to acquire multiple 2-D images (also called projections), from multiple angles. A computer is then used to apply a tomographic reconstruction algorithm to the multiple projections, yielding a 3-D data set. This data set may then be manipulated to show thin slices along any chosen axis of the body, similar to those obtained from other tomographic techniques, such as magnetic resonance imaging (MRI), X-ray computed tomography (X-ray CT), and positron emission tomography (PET).

SPECT is similar to PET in its use of radioactive tracer material and detection of gamma rays. In contrast with PET, the tracers used in SPECT emit gamma radiation that is measured directly, whereas PET tracers emit positrons that annihilate with electrons up to a few millimeters away, causing two gamma photons to be emitted in opposite directions. A PET scanner detects these emissions "coincident" in time, which provides more radiation event localization information and, thus, higher spatial resolution images than SPECT (which has about 1 cm resolution). SPECT scans are significantly less expensive than PET scans, in part because they are able to use longer-lived and more easily obtained radioisotopes than PET.

Because SPECT acquisition is very similar to planar gamma camera imaging, the same radiopharmaceuticals may be used. If a patient is examined in another type of nuclear medicine scan, but the images are non-diagnostic, it may be possible to proceed straight to SPECT by moving the patient to a SPECT instrument, or even by simply reconfiguring the camera for SPECT image acquisition while the patient remains on the table.

\end{document}
