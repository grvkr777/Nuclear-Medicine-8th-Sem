\documentclass{article}
\usepackage[utf8]{inputenc}

\title{Update-6}
\author{Gaurav Kar}
\date{18111025}

\begin{document}

\maketitle

\section{Hybrid scanning techniques}
In some centers, the nuclear medicine scans can be superimposed, using software or hybrid cameras, on images from modalities such as CT or MRI to highlight the part of the body in which the radiopharmaceutical is concentrated. This practice is often referred to as image fusion or co-registration, for example SPECT/CT and PET/CT. The fusion imaging technique in nuclear medicine provides information about the anatomy and function, which would otherwise be unavailable or would require a more invasive procedure or surgery.\\Practical concerns in nuclear imaging
Although the risks of low-level radiation exposures are not well understood, a cautious approach has been universally adopted that all human radiation exposures should be kept As Low As Reasonably Practicable, "ALARP". (Originally, this was known as "As Low As Reasonably Achievable" (ALARA), but this has changed in modern draftings of the legislation to add more emphasis on the "Reasonably" and less on the "Achievable".)

Working with the ALARP principle, before a patient is exposed for a nuclear medicine examination, the benefit of the examination must be identified. This needs to take into account the particular circumstances of the patient in question, where appropriate. For instance, if a patient is unlikely to be able to tolerate a sufficient amount of the procedure to achieve a diagnosis, then it would be inappropriate to proceed with injecting the patient with the radioactive tracer.When the benefit does justify the procedure, then the radiation exposure (the amount of radiation given to the patient) should also be kept as low as reasonably practicable. This means that the images produced in nuclear medicine should never be better than required for confident diagnosis. Giving larger radiation exposures can reduce the noise in an image and make it more photographically appealing, but if the clinical question can be answered without this level of detail, then this is inappropriate.\\As a result, the radiation dose from nuclear medicine imaging varies greatly depending on the type of study. The effective radiation dose can be lower than or comparable to or can far exceed the general day-to-day environmental annual background radiation dose. Likewise, it can also be less than, in the range of, or higher than the radiation dose from an abdomen/pelvis CT scan.\\Some nuclear medicine procedures require special patient preparation before the study to obtain the most accurate result. Pre-imaging preparations may include dietary preparation or the withholding of certain medications. Patients are encouraged to consult with the nuclear medicine department prior to a scan.

\end{document}
