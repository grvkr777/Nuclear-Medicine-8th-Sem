\documentclass{article}
\usepackage[utf8]{inputenc}

\title{Update-2}
\author{Diagnostic Medical Imaging}
\date{Gaurav Kar (18111025)}

\begin{document}

\maketitle

\section{Continuation}
In nuclear medicine imaging, radiopharmaceuticals are taken internally, for example, through inhalation, intravenously or orally. Then, external detectors (gamma cameras) capture and form images from the radiation emitted by the radiopharmaceuticals. This process is unlike a diagnostic X-ray, where external radiation is passed through the body to form an image.There are several techniques of diagnostic nuclear medicine.\\For example: Cintigraphy (from Latin scintilla, "spark"), also known as a gamma scan, is a diagnostic test in nuclear medicine, where radioisotopes attached to drugs that travel to a specific organ or tissue (radiopharmaceuticals) are taken internally and the emitted gamma radiation is captured by external detectors (gamma cameras) to form two-dimensional images in a similar process to the capture of x-ray images. In contrast, SPECT and positron emission tomography (PET) form 3-dimensional images and are therefore classified as separate techniques from scintigraphy, although they also use gamma cameras to detect internal radiation. Scintigraphy is unlike a diagnostic X-ray where external radiation is passed through the body to form an image.\\Scintillography is an imaging method of nuclear events provoked by collisions or charged current interactions among nuclear particles or ionizing radiation and atoms which result in a brief, localised pulse of electromagnetic radiation, usually in the visible light range (Cherenkov radiation). This pulse (scintillation) is usually detected and amplified by a photomultiplier or charged coupled device elements, and its resulting electrical waveform is processed by computers to provide two- and three-dimensional images of a subject or region of interest. Scintillography is mainly used in scintillation cameras in experimental physics. For example, huge neutrino detection underground tanks filled with tetrachloroethylene are surrounded by arrays of photo detectors in order to capture the extremely rare event of a collision between the fluid's atoms and a neutrino. Another extensive use of scintillography is in medical imaging techniques which use gamma ray detectors called gamma cameras. Detectors coated with materials which scintillate when subjected to gamma rays are scanned with optical photon detectors and scintillation counters. The subjects are injected with special radionuclides which irradiate in the gamma range inside the region of interest, such as the heart or the brain. A special type of gamma camera is the SPECT (Single Photon Emission Computed Tomography). Another medical scintillography technique, the Positron-emission tomography (PET), which uses the scintillations provoked by electron-positron annihilation phenomena.
\end{document}
