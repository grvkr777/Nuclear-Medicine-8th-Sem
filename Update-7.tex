\documentclass{article}
\usepackage[utf8]{inputenc}

\title{Update-7}
\author{Gaurav Kar (18111025)}
\date{}

\begin{document}

\maketitle

\section{Continuation}
Applications in nuclear technology: In the nuclear power sector, the SPECT technique can be applied to image radioisotope distributions in irradiated nuclear fuels. Due to the irradiation of nuclear fuel (e.g. uranium) with neutrons in a nuclear reactor, a wide array of gamma-emitting radionuclides are naturally produced in the fuel, such as fission products (cesium-137, barium-140 and europium-154) and activation products (chromium-51 and cobalt-58). These may be imaged using SPECT in order to verify the presence of fuel rods in a stored fuel assembly for IAEA safeguards purposes,[8] to validate predictions of core simulation codes, or to study the behavior of the nuclear fuel in normal operation, or in accident scenarios.\\Myocardial perfusion imaging:Myocardial perfusion imaging (MPI) is a form of functional cardiac imaging, used for the diagnosis of ischemic heart disease. The underlying principle is that under conditions of stress, diseased myocardium receives less blood flow than normal myocardium. MPI is one of several types of cardiac stress test. A cardiac specific radiopharmaceutical is administered, e.g., 99mTc-tetrofosmin (Myoview, GE healthcare), 99mTc-sestamibi (Cardiolite, Bristol-Myers Squibb) or Thallium-201 chloride. Following this, the heart rate is raised to induce myocardial stress, either by exercise on a treadmill or pharmacologically with adenosine, dobutamine, or dipyridamole (aminophylline can be used to reverse the effects of dipyridamole). SPECT imaging performed after stress reveals the distribution of the radiopharmaceutical, and therefore the relative blood flow to the different regions of the myocardium. Diagnosis is made by comparing stress images to a further set of images obtained at rest which are normally acquired prior to the stress images. MPI has been demonstrated to have an overall accuracy of about 83 percent (sensitivity: 85 percent; specificity: 72 percent) (in a review, not exclusively of SPECT MPI), and is comparable with (or better than) other non-invasive tests for ischemic heart disease.
\end{document}
