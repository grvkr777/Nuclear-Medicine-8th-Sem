\documentclass{article}
\usepackage[utf8]{inputenc}

\title{Update-1}
\author{Diagnostic Medical Imaging}
\date{Gaurav Kar (18111025)}

\begin{document}

\maketitle

\section{Introduction}
Diagnostic imaging describes various techniques of viewing the inside of the body to help figure out the causes of an illness or injury and confirm a diagnosis. Doctors also use it to see how well a patient’s body responds to treatment for a fracture or illness. In nuclear medicine imaging, radiopharmaceuticals are taken internally, for example, through inhalation, intravenously or orally. Then, external detectors (gamma cameras) capture and form images from the radiation emitted by the radiopharmaceuticals. This process is unlike a diagnostic X-ray, where external radiation is passed through the body to form an image. Diagnostic imaging allows physicians to view the inside of your body to help them find any indications of a health condition. Some machines and methods can produce pictures of the activities and structures inside your body. Your doctor will decide which medical imaging tests they’ll need to use based on the body part they’re evaluating and your symptoms. Many imaging tests are noninvasive, easy and painless. Some will require you to remain still inside the machine for a long time, however, which can get a little uncomfortable. Some tests involve a small amount of radiation exposure.\\
For other imaging tests, the doctor will insert a small camera attached to a thin, long tube into your body. This device is referred to as a “scope.” They’ll then move the scope through a bodily opening or passageway to view the inside of a particular organ, like your lungs, heart or colon. You may need anesthesia for these procedures.\\Nuclear medicine imaging is a method of producing images by detecting radiation from different parts of the body after a radioactive tracer is given to the patient. The images are digitally generated on a computer and transferred to a nuclear medicine physician, who interprets the images to make a diagnosis.\\Radioactive tracers used in nuclear medicine are, in most cases, injected into a vein. For some studies, they may be given by mouth. These tracers aren’t dyes or medicines, and they have no side effects. The amount of radiation a patient receives in a typical nuclear medicine scan tends to be very low.
\end{document}
